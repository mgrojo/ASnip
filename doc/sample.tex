% Demo of ASnip using \TeX\ macros from WEB (webmac.tex).
\input pdfwebmac


%%%%%%%%%%%%%%%%
% redefine \C : remove the Pascal commment braces around the comment text
\def\C#1{\ifmmode\gdef\XX{\null$\null}\else\gdef\XX{}\fi % Pascal comments
  \XX\hfil\penalty-1\hfilneg\quad#1\XX}%$-matching

\def\title{ASnip Demo} % and we should not announce to be WEB output!

\def\beginAda{\Y\P}
\def\endAda{\par\fi}

%%%%%%%%%%%%%%%%

\N1. Hello.
ASnip can decorate Ada source text borrowing macros from the
standard \TeX\ file {\tt webmac.tex}. This file is part of the \TeX\
distribution, for use with the {\tt weave} program, which translates
literate programs written using Knuth's Web System of Structured
Documentation.



\M2. The following two programs have been decorated by ASnip, reading
the very same source files that are used when making the HTML examples.
First, Hello:

\beginAda
\input hello_body.in
\endAda

\M3. The SETL/2 program, seen through ASnip's Ada glasses.
Again, ``{\tt program}'' is not a reserved word in Ada.

\beginAda
\input quicksort_body.in
\endAda

\bye
